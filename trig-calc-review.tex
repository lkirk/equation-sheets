\documentclass[twocolumn]{amsart}
\usepackage[pdftex, a4paper, margin=0.7cm, nohead, nofoot]{geometry}
% \usepackage[utopia]{mathdesign}
\usepackage{amsmath}
\usepackage{amsfonts}
\usepackage{bbm}
\usepackage{setspace}
\usepackage{scalefnt}
\usepackage{microtype}
\usepackage[absolute]{textpos}
\usepackage[compact]{titlesec}

\setlength{\parskip}{0pt}
\setlength{\parindent}{0pt}
\setlength{\TPHorizModule}{30mm}
\setlength{\TPVertModule}{\TPHorizModule}
\textblockorigin{10mm}{10mm} % start everything near the top-left corner
\titleformat{\section}{\centering\selectfont\bf}{}{0em}{}

\def\parsedate #1:20#2#3#4#5#6#7#8\empty{#6#7/#4#5/20#2#3\parsetime#8\empty}
\def\parsetime #1#2#3#4#5\empty{ #1#2:#3#4}
\def\moddate#1{\expandafter\parsedate\pdffilemoddate{#1}\empty}

% template taken from here: https://github.com/daleroberts/math-finance-cheat-sheet/

% More probability stuff
% https://wjgan.com/posts/latex.html

\begin{document}
\pagestyle{empty}
\thispagestyle{empty}
\setstretch{0.8}
\scalefont{0.8}

\begin{center}
\textbf{Trig Calc Review}
\vskip0.2em
\end{center}

\section*{Derivatives}

\begin{equation*}
\frac{\partial \sin (x)}{\partial x} = \cos (x)
\end{equation*}
\begin{equation*}
\frac{\partial \sin ^{-1}(x)}{\partial x} = \frac{1}{\sqrt{1-x^2}}
\end{equation*}
\begin{equation*}
\frac{\partial \sinh (x)}{\partial x} = \cosh (x)
\end{equation*}
\begin{equation*}
\frac{\partial \sinh ^{-1}(x)}{\partial x} = \frac{1}{\sqrt{x^2+1}}
\end{equation*}
\begin{equation*}
\frac{\partial \cos (x)}{\partial x} = -\sin (x)
\end{equation*}
\begin{equation*}
\frac{\partial \cosh (x)}{\partial x} = \sinh (x)
\end{equation*}
\begin{equation*}
\frac{\partial \cos ^{-1}(x)}{\partial x} = -\frac{1}{\sqrt{1-x^2}}
\end{equation*}
\begin{equation*}
\frac{\partial \cosh ^{-1}(x)}{\partial x} = \frac{1}{\sqrt{x-1} \sqrt{x+1}}
\end{equation*}
\begin{equation*}
\frac{\partial \tan (x)}{\partial x} = \sec ^2(x)
\end{equation*}
\begin{equation*}
\frac{\partial \tan ^{-1}(x)}{\partial x} = \frac{1}{x^2+1}
\end{equation*}
\begin{equation*}
\frac{\partial \tanh (x)}{\partial x} = \text{sech}^2(x)
\end{equation*}
\begin{equation*}
\frac{\partial \tanh ^{-1}(x)}{\partial x} = \frac{1}{1-x^2}
\end{equation*}

\section*{Integrals}

\begin{equation*}  
\int \sin (x) \, dx = -\cos (x)
\end{equation*}

\begin{equation*}
\int \sin ^{-1}(x) \, dx = \sqrt{1-x^2}+x \sin ^{-1}(x)
\end{equation*}
\begin{equation*}
\int \sinh (x) \, dx = \cosh (x)
\end{equation*}
\begin{equation*}
\int \sinh ^{-1}(x) \, dx = x \sinh ^{-1}(x)-\sqrt{x^2+1}
\end{equation*}
\begin{equation*}
\int \cos (x) \, dx = \sin (x)
\end{equation*}
\begin{equation*}
\int \cosh (x) \, dx = \sinh (x)
\end{equation*}
\begin{equation*}
\int \cos ^{-1}(x) \, dx = x \cos ^{-1}(x)-\sqrt{1-x^2}
\end{equation*}
\begin{equation*}
\int \cosh ^{-1}(x) \, dx = x \cosh ^{-1}(x)-\sqrt{x-1} \sqrt{x+1}
\end{equation*}
\begin{equation*}
\int \tan (x) \, dx = -\log (\cos (x))
\end{equation*}
\begin{equation*}
\int \tan ^{-1}(x) \, dx = x \tan ^{-1}(x)-\frac{1}{2} \log \left(x^2+1\right)
\end{equation*}
\begin{equation*}
\int \tanh (x) \, dx = \log (\cosh (x))
\end{equation*}
\begin{equation*}
\int \tanh ^{-1}(x) \, dx = \frac{1}{2} \log \left(1-x^2\right)+x \tanh ^{-1}(x)
\end{equation*}


\end{document}